\documentclass[11pt]{article}

    \usepackage[breakable]{tcolorbox}
    \usepackage{parskip} % Stop auto-indenting (to mimic markdown behaviour)
    
    \usepackage{iftex}
    \ifPDFTeX
    	\usepackage[T1]{fontenc}
    	\usepackage{mathpazo}
    \else
    	\usepackage{fontspec}
    \fi

    % Basic figure setup, for now with no caption control since it's done
    % automatically by Pandoc (which extracts ![](path) syntax from Markdown).
    \usepackage{graphicx}
    % Maintain compatibility with old templates. Remove in nbconvert 6.0
    \let\Oldincludegraphics\includegraphics
    % Ensure that by default, figures have no caption (until we provide a
    % proper Figure object with a Caption API and a way to capture that
    % in the conversion process - todo).
    \usepackage{caption}
    \DeclareCaptionFormat{nocaption}{}
    \captionsetup{format=nocaption,aboveskip=0pt,belowskip=0pt}

    \usepackage{float}
    \floatplacement{figure}{H} % forces figures to be placed at the correct location
    \usepackage{xcolor} % Allow colors to be defined
    \usepackage{enumerate} % Needed for markdown enumerations to work
    \usepackage{geometry} % Used to adjust the document margins
    \usepackage{amsmath} % Equations
    \usepackage{amssymb} % Equations
    \usepackage{textcomp} % defines textquotesingle
    % Hack from http://tex.stackexchange.com/a/47451/13684:
    \AtBeginDocument{%
        \def\PYZsq{\textquotesingle}% Upright quotes in Pygmentized code
    }
    \usepackage{upquote} % Upright quotes for verbatim code
    \usepackage{eurosym} % defines \euro
    \usepackage[mathletters]{ucs} % Extended unicode (utf-8) support
    \usepackage{fancyvrb} % verbatim replacement that allows latex
    \usepackage{grffile} % extends the file name processing of package graphics 
                         % to support a larger range
    \makeatletter % fix for old versions of grffile with XeLaTeX
    \@ifpackagelater{grffile}{2019/11/01}
    {
      % Do nothing on new versions
    }
    {
      \def\Gread@@xetex#1{%
        \IfFileExists{"\Gin@base".bb}%
        {\Gread@eps{\Gin@base.bb}}%
        {\Gread@@xetex@aux#1}%
      }
    }
    \makeatother
    \usepackage[Export]{adjustbox} % Used to constrain images to a maximum size
    \adjustboxset{max size={0.9\linewidth}{0.9\paperheight}}

    % The hyperref package gives us a pdf with properly built
    % internal navigation ('pdf bookmarks' for the table of contents,
    % internal cross-reference links, web links for URLs, etc.)
    \usepackage{hyperref}
    % The default LaTeX title has an obnoxious amount of whitespace. By default,
    % titling removes some of it. It also provides customization options.
    \usepackage{titling}
    \usepackage{longtable} % longtable support required by pandoc >1.10
    \usepackage{booktabs}  % table support for pandoc > 1.12.2
    \usepackage[inline]{enumitem} % IRkernel/repr support (it uses the enumerate* environment)
    \usepackage[normalem]{ulem} % ulem is needed to support strikethroughs (\sout)
                                % normalem makes italics be italics, not underlines
    \usepackage{mathrsfs}
    

    
    % Colors for the hyperref package
    \definecolor{urlcolor}{rgb}{0,.145,.698}
    \definecolor{linkcolor}{rgb}{.71,0.21,0.01}
    \definecolor{citecolor}{rgb}{.12,.54,.11}

    % ANSI colors
    \definecolor{ansi-black}{HTML}{3E424D}
    \definecolor{ansi-black-intense}{HTML}{282C36}
    \definecolor{ansi-red}{HTML}{E75C58}
    \definecolor{ansi-red-intense}{HTML}{B22B31}
    \definecolor{ansi-green}{HTML}{00A250}
    \definecolor{ansi-green-intense}{HTML}{007427}
    \definecolor{ansi-yellow}{HTML}{DDB62B}
    \definecolor{ansi-yellow-intense}{HTML}{B27D12}
    \definecolor{ansi-blue}{HTML}{208FFB}
    \definecolor{ansi-blue-intense}{HTML}{0065CA}
    \definecolor{ansi-magenta}{HTML}{D160C4}
    \definecolor{ansi-magenta-intense}{HTML}{A03196}
    \definecolor{ansi-cyan}{HTML}{60C6C8}
    \definecolor{ansi-cyan-intense}{HTML}{258F8F}
    \definecolor{ansi-white}{HTML}{C5C1B4}
    \definecolor{ansi-white-intense}{HTML}{A1A6B2}
    \definecolor{ansi-default-inverse-fg}{HTML}{FFFFFF}
    \definecolor{ansi-default-inverse-bg}{HTML}{000000}

    % common color for the border for error outputs.
    \definecolor{outerrorbackground}{HTML}{FFDFDF}

    % commands and environments needed by pandoc snippets
    % extracted from the output of `pandoc -s`
    \providecommand{\tightlist}{%
      \setlength{\itemsep}{0pt}\setlength{\parskip}{0pt}}
    \DefineVerbatimEnvironment{Highlighting}{Verbatim}{commandchars=\\\{\}}
    % Add ',fontsize=\small' for more characters per line
    \newenvironment{Shaded}{}{}
    \newcommand{\KeywordTok}[1]{\textcolor[rgb]{0.00,0.44,0.13}{\textbf{{#1}}}}
    \newcommand{\DataTypeTok}[1]{\textcolor[rgb]{0.56,0.13,0.00}{{#1}}}
    \newcommand{\DecValTok}[1]{\textcolor[rgb]{0.25,0.63,0.44}{{#1}}}
    \newcommand{\BaseNTok}[1]{\textcolor[rgb]{0.25,0.63,0.44}{{#1}}}
    \newcommand{\FloatTok}[1]{\textcolor[rgb]{0.25,0.63,0.44}{{#1}}}
    \newcommand{\CharTok}[1]{\textcolor[rgb]{0.25,0.44,0.63}{{#1}}}
    \newcommand{\StringTok}[1]{\textcolor[rgb]{0.25,0.44,0.63}{{#1}}}
    \newcommand{\CommentTok}[1]{\textcolor[rgb]{0.38,0.63,0.69}{\textit{{#1}}}}
    \newcommand{\OtherTok}[1]{\textcolor[rgb]{0.00,0.44,0.13}{{#1}}}
    \newcommand{\AlertTok}[1]{\textcolor[rgb]{1.00,0.00,0.00}{\textbf{{#1}}}}
    \newcommand{\FunctionTok}[1]{\textcolor[rgb]{0.02,0.16,0.49}{{#1}}}
    \newcommand{\RegionMarkerTok}[1]{{#1}}
    \newcommand{\ErrorTok}[1]{\textcolor[rgb]{1.00,0.00,0.00}{\textbf{{#1}}}}
    \newcommand{\NormalTok}[1]{{#1}}
    
    % Additional commands for more recent versions of Pandoc
    \newcommand{\ConstantTok}[1]{\textcolor[rgb]{0.53,0.00,0.00}{{#1}}}
    \newcommand{\SpecialCharTok}[1]{\textcolor[rgb]{0.25,0.44,0.63}{{#1}}}
    \newcommand{\VerbatimStringTok}[1]{\textcolor[rgb]{0.25,0.44,0.63}{{#1}}}
    \newcommand{\SpecialStringTok}[1]{\textcolor[rgb]{0.73,0.40,0.53}{{#1}}}
    \newcommand{\ImportTok}[1]{{#1}}
    \newcommand{\DocumentationTok}[1]{\textcolor[rgb]{0.73,0.13,0.13}{\textit{{#1}}}}
    \newcommand{\AnnotationTok}[1]{\textcolor[rgb]{0.38,0.63,0.69}{\textbf{\textit{{#1}}}}}
    \newcommand{\CommentVarTok}[1]{\textcolor[rgb]{0.38,0.63,0.69}{\textbf{\textit{{#1}}}}}
    \newcommand{\VariableTok}[1]{\textcolor[rgb]{0.10,0.09,0.49}{{#1}}}
    \newcommand{\ControlFlowTok}[1]{\textcolor[rgb]{0.00,0.44,0.13}{\textbf{{#1}}}}
    \newcommand{\OperatorTok}[1]{\textcolor[rgb]{0.40,0.40,0.40}{{#1}}}
    \newcommand{\BuiltInTok}[1]{{#1}}
    \newcommand{\ExtensionTok}[1]{{#1}}
    \newcommand{\PreprocessorTok}[1]{\textcolor[rgb]{0.74,0.48,0.00}{{#1}}}
    \newcommand{\AttributeTok}[1]{\textcolor[rgb]{0.49,0.56,0.16}{{#1}}}
    \newcommand{\InformationTok}[1]{\textcolor[rgb]{0.38,0.63,0.69}{\textbf{\textit{{#1}}}}}
    \newcommand{\WarningTok}[1]{\textcolor[rgb]{0.38,0.63,0.69}{\textbf{\textit{{#1}}}}}
    
    
    % Define a nice break command that doesn't care if a line doesn't already
    % exist.
    \def\br{\hspace*{\fill} \\* }
    % Math Jax compatibility definitions
    \def\gt{>}
    \def\lt{<}
    \let\Oldtex\TeX
    \let\Oldlatex\LaTeX
    \renewcommand{\TeX}{\textrm{\Oldtex}}
    \renewcommand{\LaTeX}{\textrm{\Oldlatex}}
    % Document parameters
    % Document title
    \title{feature\_report}
    
    
    
    
    
% Pygments definitions
\makeatletter
\def\PY@reset{\let\PY@it=\relax \let\PY@bf=\relax%
    \let\PY@ul=\relax \let\PY@tc=\relax%
    \let\PY@bc=\relax \let\PY@ff=\relax}
\def\PY@tok#1{\csname PY@tok@#1\endcsname}
\def\PY@toks#1+{\ifx\relax#1\empty\else%
    \PY@tok{#1}\expandafter\PY@toks\fi}
\def\PY@do#1{\PY@bc{\PY@tc{\PY@ul{%
    \PY@it{\PY@bf{\PY@ff{#1}}}}}}}
\def\PY#1#2{\PY@reset\PY@toks#1+\relax+\PY@do{#2}}

\expandafter\def\csname PY@tok@w\endcsname{\def\PY@tc##1{\textcolor[rgb]{0.73,0.73,0.73}{##1}}}
\expandafter\def\csname PY@tok@c\endcsname{\let\PY@it=\textit\def\PY@tc##1{\textcolor[rgb]{0.25,0.50,0.50}{##1}}}
\expandafter\def\csname PY@tok@cp\endcsname{\def\PY@tc##1{\textcolor[rgb]{0.74,0.48,0.00}{##1}}}
\expandafter\def\csname PY@tok@k\endcsname{\let\PY@bf=\textbf\def\PY@tc##1{\textcolor[rgb]{0.00,0.50,0.00}{##1}}}
\expandafter\def\csname PY@tok@kp\endcsname{\def\PY@tc##1{\textcolor[rgb]{0.00,0.50,0.00}{##1}}}
\expandafter\def\csname PY@tok@kt\endcsname{\def\PY@tc##1{\textcolor[rgb]{0.69,0.00,0.25}{##1}}}
\expandafter\def\csname PY@tok@o\endcsname{\def\PY@tc##1{\textcolor[rgb]{0.40,0.40,0.40}{##1}}}
\expandafter\def\csname PY@tok@ow\endcsname{\let\PY@bf=\textbf\def\PY@tc##1{\textcolor[rgb]{0.67,0.13,1.00}{##1}}}
\expandafter\def\csname PY@tok@nb\endcsname{\def\PY@tc##1{\textcolor[rgb]{0.00,0.50,0.00}{##1}}}
\expandafter\def\csname PY@tok@nf\endcsname{\def\PY@tc##1{\textcolor[rgb]{0.00,0.00,1.00}{##1}}}
\expandafter\def\csname PY@tok@nc\endcsname{\let\PY@bf=\textbf\def\PY@tc##1{\textcolor[rgb]{0.00,0.00,1.00}{##1}}}
\expandafter\def\csname PY@tok@nn\endcsname{\let\PY@bf=\textbf\def\PY@tc##1{\textcolor[rgb]{0.00,0.00,1.00}{##1}}}
\expandafter\def\csname PY@tok@ne\endcsname{\let\PY@bf=\textbf\def\PY@tc##1{\textcolor[rgb]{0.82,0.25,0.23}{##1}}}
\expandafter\def\csname PY@tok@nv\endcsname{\def\PY@tc##1{\textcolor[rgb]{0.10,0.09,0.49}{##1}}}
\expandafter\def\csname PY@tok@no\endcsname{\def\PY@tc##1{\textcolor[rgb]{0.53,0.00,0.00}{##1}}}
\expandafter\def\csname PY@tok@nl\endcsname{\def\PY@tc##1{\textcolor[rgb]{0.63,0.63,0.00}{##1}}}
\expandafter\def\csname PY@tok@ni\endcsname{\let\PY@bf=\textbf\def\PY@tc##1{\textcolor[rgb]{0.60,0.60,0.60}{##1}}}
\expandafter\def\csname PY@tok@na\endcsname{\def\PY@tc##1{\textcolor[rgb]{0.49,0.56,0.16}{##1}}}
\expandafter\def\csname PY@tok@nt\endcsname{\let\PY@bf=\textbf\def\PY@tc##1{\textcolor[rgb]{0.00,0.50,0.00}{##1}}}
\expandafter\def\csname PY@tok@nd\endcsname{\def\PY@tc##1{\textcolor[rgb]{0.67,0.13,1.00}{##1}}}
\expandafter\def\csname PY@tok@s\endcsname{\def\PY@tc##1{\textcolor[rgb]{0.73,0.13,0.13}{##1}}}
\expandafter\def\csname PY@tok@sd\endcsname{\let\PY@it=\textit\def\PY@tc##1{\textcolor[rgb]{0.73,0.13,0.13}{##1}}}
\expandafter\def\csname PY@tok@si\endcsname{\let\PY@bf=\textbf\def\PY@tc##1{\textcolor[rgb]{0.73,0.40,0.53}{##1}}}
\expandafter\def\csname PY@tok@se\endcsname{\let\PY@bf=\textbf\def\PY@tc##1{\textcolor[rgb]{0.73,0.40,0.13}{##1}}}
\expandafter\def\csname PY@tok@sr\endcsname{\def\PY@tc##1{\textcolor[rgb]{0.73,0.40,0.53}{##1}}}
\expandafter\def\csname PY@tok@ss\endcsname{\def\PY@tc##1{\textcolor[rgb]{0.10,0.09,0.49}{##1}}}
\expandafter\def\csname PY@tok@sx\endcsname{\def\PY@tc##1{\textcolor[rgb]{0.00,0.50,0.00}{##1}}}
\expandafter\def\csname PY@tok@m\endcsname{\def\PY@tc##1{\textcolor[rgb]{0.40,0.40,0.40}{##1}}}
\expandafter\def\csname PY@tok@gh\endcsname{\let\PY@bf=\textbf\def\PY@tc##1{\textcolor[rgb]{0.00,0.00,0.50}{##1}}}
\expandafter\def\csname PY@tok@gu\endcsname{\let\PY@bf=\textbf\def\PY@tc##1{\textcolor[rgb]{0.50,0.00,0.50}{##1}}}
\expandafter\def\csname PY@tok@gd\endcsname{\def\PY@tc##1{\textcolor[rgb]{0.63,0.00,0.00}{##1}}}
\expandafter\def\csname PY@tok@gi\endcsname{\def\PY@tc##1{\textcolor[rgb]{0.00,0.63,0.00}{##1}}}
\expandafter\def\csname PY@tok@gr\endcsname{\def\PY@tc##1{\textcolor[rgb]{1.00,0.00,0.00}{##1}}}
\expandafter\def\csname PY@tok@ge\endcsname{\let\PY@it=\textit}
\expandafter\def\csname PY@tok@gs\endcsname{\let\PY@bf=\textbf}
\expandafter\def\csname PY@tok@gp\endcsname{\let\PY@bf=\textbf\def\PY@tc##1{\textcolor[rgb]{0.00,0.00,0.50}{##1}}}
\expandafter\def\csname PY@tok@go\endcsname{\def\PY@tc##1{\textcolor[rgb]{0.53,0.53,0.53}{##1}}}
\expandafter\def\csname PY@tok@gt\endcsname{\def\PY@tc##1{\textcolor[rgb]{0.00,0.27,0.87}{##1}}}
\expandafter\def\csname PY@tok@err\endcsname{\def\PY@bc##1{\setlength{\fboxsep}{0pt}\fcolorbox[rgb]{1.00,0.00,0.00}{1,1,1}{\strut ##1}}}
\expandafter\def\csname PY@tok@kc\endcsname{\let\PY@bf=\textbf\def\PY@tc##1{\textcolor[rgb]{0.00,0.50,0.00}{##1}}}
\expandafter\def\csname PY@tok@kd\endcsname{\let\PY@bf=\textbf\def\PY@tc##1{\textcolor[rgb]{0.00,0.50,0.00}{##1}}}
\expandafter\def\csname PY@tok@kn\endcsname{\let\PY@bf=\textbf\def\PY@tc##1{\textcolor[rgb]{0.00,0.50,0.00}{##1}}}
\expandafter\def\csname PY@tok@kr\endcsname{\let\PY@bf=\textbf\def\PY@tc##1{\textcolor[rgb]{0.00,0.50,0.00}{##1}}}
\expandafter\def\csname PY@tok@bp\endcsname{\def\PY@tc##1{\textcolor[rgb]{0.00,0.50,0.00}{##1}}}
\expandafter\def\csname PY@tok@fm\endcsname{\def\PY@tc##1{\textcolor[rgb]{0.00,0.00,1.00}{##1}}}
\expandafter\def\csname PY@tok@vc\endcsname{\def\PY@tc##1{\textcolor[rgb]{0.10,0.09,0.49}{##1}}}
\expandafter\def\csname PY@tok@vg\endcsname{\def\PY@tc##1{\textcolor[rgb]{0.10,0.09,0.49}{##1}}}
\expandafter\def\csname PY@tok@vi\endcsname{\def\PY@tc##1{\textcolor[rgb]{0.10,0.09,0.49}{##1}}}
\expandafter\def\csname PY@tok@vm\endcsname{\def\PY@tc##1{\textcolor[rgb]{0.10,0.09,0.49}{##1}}}
\expandafter\def\csname PY@tok@sa\endcsname{\def\PY@tc##1{\textcolor[rgb]{0.73,0.13,0.13}{##1}}}
\expandafter\def\csname PY@tok@sb\endcsname{\def\PY@tc##1{\textcolor[rgb]{0.73,0.13,0.13}{##1}}}
\expandafter\def\csname PY@tok@sc\endcsname{\def\PY@tc##1{\textcolor[rgb]{0.73,0.13,0.13}{##1}}}
\expandafter\def\csname PY@tok@dl\endcsname{\def\PY@tc##1{\textcolor[rgb]{0.73,0.13,0.13}{##1}}}
\expandafter\def\csname PY@tok@s2\endcsname{\def\PY@tc##1{\textcolor[rgb]{0.73,0.13,0.13}{##1}}}
\expandafter\def\csname PY@tok@sh\endcsname{\def\PY@tc##1{\textcolor[rgb]{0.73,0.13,0.13}{##1}}}
\expandafter\def\csname PY@tok@s1\endcsname{\def\PY@tc##1{\textcolor[rgb]{0.73,0.13,0.13}{##1}}}
\expandafter\def\csname PY@tok@mb\endcsname{\def\PY@tc##1{\textcolor[rgb]{0.40,0.40,0.40}{##1}}}
\expandafter\def\csname PY@tok@mf\endcsname{\def\PY@tc##1{\textcolor[rgb]{0.40,0.40,0.40}{##1}}}
\expandafter\def\csname PY@tok@mh\endcsname{\def\PY@tc##1{\textcolor[rgb]{0.40,0.40,0.40}{##1}}}
\expandafter\def\csname PY@tok@mi\endcsname{\def\PY@tc##1{\textcolor[rgb]{0.40,0.40,0.40}{##1}}}
\expandafter\def\csname PY@tok@il\endcsname{\def\PY@tc##1{\textcolor[rgb]{0.40,0.40,0.40}{##1}}}
\expandafter\def\csname PY@tok@mo\endcsname{\def\PY@tc##1{\textcolor[rgb]{0.40,0.40,0.40}{##1}}}
\expandafter\def\csname PY@tok@ch\endcsname{\let\PY@it=\textit\def\PY@tc##1{\textcolor[rgb]{0.25,0.50,0.50}{##1}}}
\expandafter\def\csname PY@tok@cm\endcsname{\let\PY@it=\textit\def\PY@tc##1{\textcolor[rgb]{0.25,0.50,0.50}{##1}}}
\expandafter\def\csname PY@tok@cpf\endcsname{\let\PY@it=\textit\def\PY@tc##1{\textcolor[rgb]{0.25,0.50,0.50}{##1}}}
\expandafter\def\csname PY@tok@c1\endcsname{\let\PY@it=\textit\def\PY@tc##1{\textcolor[rgb]{0.25,0.50,0.50}{##1}}}
\expandafter\def\csname PY@tok@cs\endcsname{\let\PY@it=\textit\def\PY@tc##1{\textcolor[rgb]{0.25,0.50,0.50}{##1}}}

\def\PYZbs{\char`\\}
\def\PYZus{\char`\_}
\def\PYZob{\char`\{}
\def\PYZcb{\char`\}}
\def\PYZca{\char`\^}
\def\PYZam{\char`\&}
\def\PYZlt{\char`\<}
\def\PYZgt{\char`\>}
\def\PYZsh{\char`\#}
\def\PYZpc{\char`\%}
\def\PYZdl{\char`\$}
\def\PYZhy{\char`\-}
\def\PYZsq{\char`\'}
\def\PYZdq{\char`\"}
\def\PYZti{\char`\~}
% for compatibility with earlier versions
\def\PYZat{@}
\def\PYZlb{[}
\def\PYZrb{]}
\makeatother


    % For linebreaks inside Verbatim environment from package fancyvrb. 
    \makeatletter
        \newbox\Wrappedcontinuationbox 
        \newbox\Wrappedvisiblespacebox 
        \newcommand*\Wrappedvisiblespace {\textcolor{red}{\textvisiblespace}} 
        \newcommand*\Wrappedcontinuationsymbol {\textcolor{red}{\llap{\tiny$\m@th\hookrightarrow$}}} 
        \newcommand*\Wrappedcontinuationindent {3ex } 
        \newcommand*\Wrappedafterbreak {\kern\Wrappedcontinuationindent\copy\Wrappedcontinuationbox} 
        % Take advantage of the already applied Pygments mark-up to insert 
        % potential linebreaks for TeX processing. 
        %        {, <, #, %, $, ' and ": go to next line. 
        %        _, }, ^, &, >, - and ~: stay at end of broken line. 
        % Use of \textquotesingle for straight quote. 
        \newcommand*\Wrappedbreaksatspecials {% 
            \def\PYGZus{\discretionary{\char`\_}{\Wrappedafterbreak}{\char`\_}}% 
            \def\PYGZob{\discretionary{}{\Wrappedafterbreak\char`\{}{\char`\{}}% 
            \def\PYGZcb{\discretionary{\char`\}}{\Wrappedafterbreak}{\char`\}}}% 
            \def\PYGZca{\discretionary{\char`\^}{\Wrappedafterbreak}{\char`\^}}% 
            \def\PYGZam{\discretionary{\char`\&}{\Wrappedafterbreak}{\char`\&}}% 
            \def\PYGZlt{\discretionary{}{\Wrappedafterbreak\char`\<}{\char`\<}}% 
            \def\PYGZgt{\discretionary{\char`\>}{\Wrappedafterbreak}{\char`\>}}% 
            \def\PYGZsh{\discretionary{}{\Wrappedafterbreak\char`\#}{\char`\#}}% 
            \def\PYGZpc{\discretionary{}{\Wrappedafterbreak\char`\%}{\char`\%}}% 
            \def\PYGZdl{\discretionary{}{\Wrappedafterbreak\char`\$}{\char`\$}}% 
            \def\PYGZhy{\discretionary{\char`\-}{\Wrappedafterbreak}{\char`\-}}% 
            \def\PYGZsq{\discretionary{}{\Wrappedafterbreak\textquotesingle}{\textquotesingle}}% 
            \def\PYGZdq{\discretionary{}{\Wrappedafterbreak\char`\"}{\char`\"}}% 
            \def\PYGZti{\discretionary{\char`\~}{\Wrappedafterbreak}{\char`\~}}% 
        } 
        % Some characters . , ; ? ! / are not pygmentized. 
        % This macro makes them "active" and they will insert potential linebreaks 
        \newcommand*\Wrappedbreaksatpunct {% 
            \lccode`\~`\.\lowercase{\def~}{\discretionary{\hbox{\char`\.}}{\Wrappedafterbreak}{\hbox{\char`\.}}}% 
            \lccode`\~`\,\lowercase{\def~}{\discretionary{\hbox{\char`\,}}{\Wrappedafterbreak}{\hbox{\char`\,}}}% 
            \lccode`\~`\;\lowercase{\def~}{\discretionary{\hbox{\char`\;}}{\Wrappedafterbreak}{\hbox{\char`\;}}}% 
            \lccode`\~`\:\lowercase{\def~}{\discretionary{\hbox{\char`\:}}{\Wrappedafterbreak}{\hbox{\char`\:}}}% 
            \lccode`\~`\?\lowercase{\def~}{\discretionary{\hbox{\char`\?}}{\Wrappedafterbreak}{\hbox{\char`\?}}}% 
            \lccode`\~`\!\lowercase{\def~}{\discretionary{\hbox{\char`\!}}{\Wrappedafterbreak}{\hbox{\char`\!}}}% 
            \lccode`\~`\/\lowercase{\def~}{\discretionary{\hbox{\char`\/}}{\Wrappedafterbreak}{\hbox{\char`\/}}}% 
            \catcode`\.\active
            \catcode`\,\active 
            \catcode`\;\active
            \catcode`\:\active
            \catcode`\?\active
            \catcode`\!\active
            \catcode`\/\active 
            \lccode`\~`\~ 	
        }
    \makeatother

    \let\OriginalVerbatim=\Verbatim
    \makeatletter
    \renewcommand{\Verbatim}[1][1]{%
        %\parskip\z@skip
        \sbox\Wrappedcontinuationbox {\Wrappedcontinuationsymbol}%
        \sbox\Wrappedvisiblespacebox {\FV@SetupFont\Wrappedvisiblespace}%
        \def\FancyVerbFormatLine ##1{\hsize\linewidth
            \vtop{\raggedright\hyphenpenalty\z@\exhyphenpenalty\z@
                \doublehyphendemerits\z@\finalhyphendemerits\z@
                \strut ##1\strut}%
        }%
        % If the linebreak is at a space, the latter will be displayed as visible
        % space at end of first line, and a continuation symbol starts next line.
        % Stretch/shrink are however usually zero for typewriter font.
        \def\FV@Space {%
            \nobreak\hskip\z@ plus\fontdimen3\font minus\fontdimen4\font
            \discretionary{\copy\Wrappedvisiblespacebox}{\Wrappedafterbreak}
            {\kern\fontdimen2\font}%
        }%
        
        % Allow breaks at special characters using \PYG... macros.
        \Wrappedbreaksatspecials
        % Breaks at punctuation characters . , ; ? ! and / need catcode=\active 	
        \OriginalVerbatim[#1,codes*=\Wrappedbreaksatpunct]%
    }
    \makeatother

    % Exact colors from NB
    \definecolor{incolor}{HTML}{303F9F}
    \definecolor{outcolor}{HTML}{D84315}
    \definecolor{cellborder}{HTML}{CFCFCF}
    \definecolor{cellbackground}{HTML}{F7F7F7}
    
    % prompt
    \makeatletter
    \newcommand{\boxspacing}{\kern\kvtcb@left@rule\kern\kvtcb@boxsep}
    \makeatother
    \newcommand{\prompt}[4]{
        {\ttfamily\llap{{\color{#2}[#3]:\hspace{3pt}#4}}\vspace{-\baselineskip}}
    }
    

    
    % Prevent overflowing lines due to hard-to-break entities
    \sloppy 
    % Setup hyperref package
    \hypersetup{
      breaklinks=true,  % so long urls are correctly broken across lines
      colorlinks=true,
      urlcolor=urlcolor,
      linkcolor=linkcolor,
      citecolor=citecolor,
      }
    % Slightly bigger margins than the latex defaults
    
    \geometry{verbose,tmargin=1in,bmargin=1in,lmargin=1in,rmargin=1in}
    
    

\begin{document}
    
    \maketitle
    
    

    
    \#\# Identification, quantification and analysis of observable
anthropogenic debris along swiss river and lakes (IQASL)

    IQASL is a project sponosored by the Swiss Federal Office for the
environment to quantify shoreline trash along swiss lakes and rivers in
the Rhone, Aare, Ticino and Linth/Limmat catchment areas. This is
accomplished by conducting multiple small scale and discrete
\textbf{litter surveys} throughout the river bassin. The majority of
samples are taken from lakes.

\textbf{What is a litter survey?}

A litter survey is the \textbf{identification and count of all objects
found within a delimited area}, in this study all surveys were bordered
on one side by water. Each object is placed into one of 260 categories¹.
The location, date, survey dimensions and the total number of objects in
each category is noted.

\textbf{Purpose of the surveys}

The survey results help ALL stakeholders identify the items that make up
the mass of trash found in the natural environment on the shores of
Swiss lakes and rivers. The surveys answer the following questions:

\begin{itemize}
\tightlist
\item
  What items are found?
\item
  How much is found ? (total weights and item counts)
\item
  How often are these items found?
\item
  Where do you find the most?
\end{itemize}

These are the most frequently asked questions and should be considered
when determining any mitigation or reduction strategies.

The project is based on the following assumptions:

\begin{itemize}
\tightlist
\item
  The more trash there is on the ground the more a person is likely to
  find
\item
  The survey results represent the minimum amount of trash at that site²
\item
  For each survey: finding one item does not effect the chance of
  finding another³
\end{itemize}

\textbf{Purpose of this report}

Summarize the results for the survey area and define the magnitude of
those results with respect to other survey areas.

\textbf{The survey results are presented as follows:}

\begin{itemize}
\tightlist
\item
  Total object count, total weight and weight of plastics
\item
  The most abundant objects from the survey area and all the other
  survey areas sorted by total object count
\item
  Pieces of trash per meter (pcs/m): the ratio of number of objects
  found to the length of the shoreline
\item
  Objects that were found in at least 50\% of the surveys
\end{itemize}

\textbf{Contents of this report}

\hyperref[scope]{Scope: description of river basin} * survey locations *
lakes and rivers * municipalities and effected population

\hyperref[aggregatedtotals]{Survey dimensions, locations, aggregated totals}
* weights and measures: cumulative * weights and measures: cumulative by
water feature * survey totals: pcs/m by date * material type: \% of
total

\hyperref[combinedtopten]{Trash removed: most abundant objects} * the
most common objects from the survey area * the most common objects from
all the survey areas * pcs/m of most common objects for all water
features in the survey area

\hyperref[frequency]{Trash removed the most often} * objects that are
found in more than 50\% of surveys * objects that are the most abundant
* objects found in less than 50\% of the surveys

\hyperref[matanduse]{Trash removed: utility} * utility classification:
\% total of all objects found

\hyperref[annex]{Annex} * effective data: report of missing records *
survey location coordinates * population profile and results by
municipality * itemized list of objects removed

\textbf{More information}

For more information about the project visit
\href{https://www.plagespropres.ch/}{project home}.

If you would like more information specific information about this
survey area please contact:

\begin{enumerate}
\def\labelenumi{\arabic{enumi}.}
\tightlist
\item
  Swiss federal office for the environment - Municipal waste section
\item
  analyst@hammerdirt.ch
\end{enumerate}

¹ \href{https://mcc.jrc.ec.europa.eu/documents/201702074014.pdf}{The EU
guide on monitoring marine litter} ² There is most likely more trash at
the survey site, but certainly not less than what was recorded.³
Independent observations :
\href{https://stats.stackexchange.com/questions/116355/what-does-independent-observations-mean}{stats
stackexchange}
 
            
    
    \hypertarget{results-bielersee}{%
\subsection{Results: Bielersee}\label{results-bielersee}}

    

    \hypertarget{scope}{%
\subsubsection{Scope}\label{scope}}

The Aare source is the Aare Glaciers in the Bernese Alps of
south-central Switzerland. The Aare is the longest river entirely within
Switzerland with a length of 295 km and drainage area of 17,779 km2.
Following the Aare Gorge, the river expands into the glacial Lake
Brienz. The Aare is canalized at Interlaken before entering Lake Thun
and exiting through the city of Thun.⁵ The river then flows northwest
surrounding the old city center of Bern on three sides. Continuing west
to Lake Wohlen Reservoir it turns north to Aarberg and is diverted west
into Lake Biel by the Hagneck Canal, one of a series of major water
corrections made in the 19th and 20th centuries connecting Neuchatel,
Biel and Morat lakes through canalization. From the upper end of Lake
Biel, at Nidau, the river exits through the Nidau-Büren Canal/Aare
Canal.⁶

The Limmat and Reuss rivers⁷, two major tributaries converge into the
Aare at the Limmatspitz between the cities of Brugg and Untersiggenthal
in Canton Aargau.⁸ The Aare river ends in the north-western region of
Koblenz, Switzerland where it joins the Rhine river which eventually
terminates in the North Sea.

{⁵ The Editors of Encyclopaedia Britannica. (1998, July 20). Aare River.
Retrieved from Britannica.com:
https://www.britannica.com/place/Aare-River ⁶ Standard Encyclopedia of
Worlds Rivers and Lakes. (1965) R.K. Gresswell ⁷ The Limmat and Reuss
are part of different survey areas ⁸ Pro Natura . (n.d.). Limmatspitz.
Retrieved from Pro Natura :
https://www.pronatura-ag.ch/de/Gebenstorf-limmatspitz}

    

    \hypertarget{survey-locations-municipalities}{%
\paragraph{Survey locations,
municipalities}\label{survey-locations-municipalities}}

            \begin{tcolorbox}[breakable, size=fbox, boxrule=.5pt, pad at break*=1mm, opacityfill=0]
\begin{Verbatim}[commandchars=\\\{\}]
<IPython.core.display.HTML object>
\end{Verbatim}
\end{tcolorbox}
         
            
    
    For the period between 2020-01 and 2021-04, 4,618 objects were removed
and identified in the course of 38 surveys. Those surveys were conducted
at 9 different locations. There are 7 different municipalities
represented in these results with a combined population of approximately
69,425.0

    
 
            
    
    \textbf{The municipalities in this report:}

Biel/Bienne, Gals, Le Landeron, Ligerz, Lüscherz, Nidau, Vinelz

    

            \begin{tcolorbox}[breakable, size=fbox, boxrule=.5pt, pad at break*=1mm, opacityfill=0]
\begin{Verbatim}[commandchars=\\\{\}]
                              time to survey  meters surveyed  m² surveyed  \textbackslash{}
location
Bielersee\_Vinelz\_FankhauserS           39.67              238        536.0
Camp des pêches                         3.33               37         93.0
Gals reserve                            5.00               38         64.0
Ligerz strand                           5.67               15         49.0
Lüscherz plage                          9.91              236        517.0
Lüscherz two                            2.25               21         21.0
Müllermatte                            59.33              512       4291.0
Nidau strand                            3.17               25        105.0
Strandboden-Biel                        7.15               93        420.0

                              total weight  plastic > 5mm weight  \textbackslash{}
location
Bielersee\_Vinelz\_FankhauserS         3.099                 1.864
Camp des pêches                      0.045                 0.035
Gals reserve                         0.430                 0.405
Ligerz strand                        0.295                 0.294
Lüscherz plage                       1.570                 0.232
Lüscherz two                         0.148                 0.100
Müllermatte                          5.588                 3.348
Nidau strand                         0.130                 0.002
Strandboden-Biel                     2.250                 0.440

                              plastic < 5mm weight  staff  help  \# samples  \textbackslash{}
location
Bielersee\_Vinelz\_FankhauserS              0.000840     12     0         12
Camp des pêches                           0.000000      1     0          1
Gals reserve                              0.000001      2     0          2
Ligerz strand                             0.000000      2     0          2
Lüscherz plage                            0.000030      4     0          4
Lüscherz two                              0.000000      1     0          1
Müllermatte                               0.001819     14     1         13
Nidau strand                              0.000236      1     0          1
Strandboden-Biel                          0.000172      2     0          2

                              labor hours  pieces of trash
location
Bielersee\_Vinelz\_FankhauserS           39              898
Camp des pêches                         3               53
Gals reserve                            5               48
Ligerz strand                           5              143
Lüscherz plage                          9              194
Lüscherz two                            2                8
Müllermatte                            59             2964
Nidau strand                            3               63
Strandboden-Biel                        7              247
\end{Verbatim}
\end{tcolorbox}
        
    \hyperref[top]{top} \#\#\# Survey dimensions, location and aggregated
totals.
 
            
    
    \hypertarget{cumulative-totals-all-data}{%
\paragraph{Cumulative totals all
data}\label{cumulative-totals-all-data}}

The cumulative results from 38 samples, weights are in kilograms, time
is in hours

    

            \begin{tcolorbox}[breakable, size=fbox, boxrule=.5pt, pad at break*=1mm, opacityfill=0]
\begin{Verbatim}[commandchars=\\\{\}]
index         \# samples pieces of trash meters surveyed m² surveyed  \textbackslash{}
summary total
Bielersee            38           4,618           1,215       6,096

index         total weight  plastic > 5mm weight  plastic < 5mm weight  \textbackslash{}
summary total
Bielersee               13                  6.72                   0.0

index          labor hours
summary total
Bielersee            132.0
\end{Verbatim}
\end{tcolorbox}
        
    \hypertarget{cumulative-totals-by-water-feature}{%
\paragraph{Cumulative totals by water
feature}\label{cumulative-totals-by-water-feature}}

            \begin{tcolorbox}[breakable, size=fbox, boxrule=.5pt, pad at break*=1mm, opacityfill=0]
\begin{Verbatim}[commandchars=\\\{\}]
                             \# samples pieces of trash meters surveyed  \textbackslash{}
location
Bielersee\_Vinelz\_FankhauserS        12             898             238
Camp des pêches                      1              53              37
Gals reserve                         2              48              38
Ligerz strand                        2             143              15
Lüscherz plage                       4             194             236
Lüscherz two                         1               8              21
Müllermatte                         13           2,964             512
Nidau strand                         1              63              25
Strandboden-Biel                     2             247              93

                             m² surveyed total weight  plastic > 5mm weight  \textbackslash{}
location
Bielersee\_Vinelz\_FankhauserS         536            3                 1.864
Camp des pêches                       93            0                 0.035
Gals reserve                          64            0                 0.405
Ligerz strand                         49            0                 0.294
Lüscherz plage                       517            1                 0.232
Lüscherz two                          21            0                 0.100
Müllermatte                        4,291            5                 3.348
Nidau strand                         105            0                 0.002
Strandboden-Biel                     420            2                 0.440

                              plastic < 5mm weight  labor hours
location
Bielersee\_Vinelz\_FankhauserS              0.000840           39
Camp des pêches                           0.000000            3
Gals reserve                              0.000001            5
Ligerz strand                             0.000000            5
Lüscherz plage                            0.000030            9
Lüscherz two                              0.000000            2
Müllermatte                               0.001819           59
Nidau strand                              0.000236            3
Strandboden-Biel                          0.000172            7
\end{Verbatim}
\end{tcolorbox}
        
    \hypertarget{survey-totals-pieces-per-meter-material-type-of-total}{%
\paragraph{Survey totals pieces per meter, material type \% of
total}\label{survey-totals-pieces-per-meter-material-type-of-total}}

    \begin{center}
    \adjustimage{max size={0.9\linewidth}{0.9\paperheight}}{feature_report_files/feature_report_21_0.png}
    \end{center}
    { \hspace*{\fill} \\}
    
    \hyperref[top]{top} \#\#\# The most abundant items
 
            
    
    Combined, the ten most abundant objects from Bielersee represent 65\% of
all objects found on Bielersee.

    

    \begin{center}
    \adjustimage{max size={0.9\linewidth}{0.9\paperheight}}{feature_report_files/feature_report_25_0.png}
    \end{center}
    { \hspace*{\fill} \\}
     
            
    
    Of the most abundant objects on Bielersee, 7 are also among the most
abundant in the Aare survey area. 1. Industrial sheeting 2. Expanded
polystyrene 3. Fragmented plastics 4. Food wrappers; candy, snack
packaging 5. Cigarette butts and filters 6. Packaging plastic nonfood or
unknown 7. Glass or ceramic drink bottles, pieces

    

    \hypertarget{the-most-abundant-objects-results-by-survey-location}{%
\paragraph{The most abundant objects: results by survey
location}\label{the-most-abundant-objects-results-by-survey-location}}

Some objects are found throughout the survey area and some of those
objects are found in all survey areas. Knowing which objects are most
abundant and what those values are helps idenitify sources or zones of
accumulation.

    \begin{center}
    \adjustimage{max size={0.9\linewidth}{0.9\paperheight}}{feature_report_files/feature_report_29_0.png}
    \end{center}
    { \hspace*{\fill} \\}
    
    \hyperref[top]{top} \#\#\# How often are these objects found?

    Some objects are found often and in elevated quantities, others are
found often and in small quantities and some times objects are found
less often but in large quantities. Knowing the diffference can help
find the sources.
 
            
    
    \textbf{Objects found in at least 50\% of surveys AND among the most
abundant objects found} Combined they had an average pieces per meter
per survey of 2.68, a min of 0.05 and max of 10.0 1. Fragmented plastics
2. Industrial sheeting 3. Food wrappers; candy, snack packaging 4.
Cigarette butts and filters 5. Glass or ceramic drink bottles, pieces 6.
Expanded polystyrene 7. Industrial pellets (nurdles) 8. Insulation:
includes spray foams and foam board 9. Cotton bud/swab sticks

    
 
            
    
    \textbf{Objects found in less than 50\% of the surveys AND among the
most abundant objects found}Combined they had an average pieces per
meter per survey of 0.15, a min of 0.02 and max of 0.39 1. Styrofoam
\textless{} 5mm

    
 
            
    
    \textbf{Objects found in at least 50\% of the surveys AND NOT among the
most abundant objects found}Combined they had an average pieces per
meter per survey of 0.56, a min of 0.02 and max of 2.86 1. Foil
wrappers, aluminum foil 2. Metal bottle caps, lids \& pull tabs from
cans 3. Other medical items (swabs, bandaging, adhesive plaster) 4.
Tobacco; plastic packaging, containers 5. Plastic construction waste 6.
Fireworks; rocket caps, exploded parts \& packaging 7. Packaging plastic
nonfood or unknown

    

    \hyperref[top]{top} \#\#\# Utility: percent of total objects collected

\textbf{Utility:} The utility type is based on the utilisation of the
object prior to it being discarded. Objects are placed into to one of
the 260 categories. Those categories are grouped according to
utilisation.

For example, a piece of plastic would be placed into the category
`Fragmented plastics', depending on its size. However, a piece of
plastic that was once a bucket and we know this because we are familiar
with either the brand or the product, is placed in a code for buckets⁸.

\begin{itemize}
\tightlist
\item
  \textbf{wastewater}: items released from water treatment plants
  includes items likely toilet flushed\\
\item
  \textbf{micro plastics (\textless{} 5mm)}: fragmented plastic items
  and pre-production plastic resins
\item
  \textbf{infrastructure}: items related to construction and maintenance
  of all infratructure items
\item
  \textbf{food and drink}: primarily single use plastic items related to
  consuming food and drinks outdoors\\
\item
  \textbf{agriculture}: primarily industrial sheeting includes
\item
  \textbf{tobacco}: primarily cigarette ends includes all smoking
  related material
\item
  \textbf{recreation}: includes fishing, hunting, boating and beach
  related objects, excludes food, drink and tobacco items\\
\item
  \textbf{packaging non food and drink}: packaging or wrapping material
  not identified as food, drink nor tobacco related\\
\item
  \textbf{plastic fragments}: foam and plastic fragments of
  indeterminate origin and use
\item
  \textbf{personal items}: accessories, hygiene and clothing related
\end{itemize}

{⁸ See the annex for the complete list of objects identified, includes
category and group classification}

    \begin{center}
    \adjustimage{max size={0.9\linewidth}{0.9\paperheight}}{feature_report_files/feature_report_41_0.png}
    \end{center}
    { \hspace*{\fill} \\}
    
    \hypertarget{more-information}{%
\subsubsection{More information}\label{more-information}}

Contact analyst@hammerdirt.ch for any questions about the content of
this report. If you would like a report for your municipality contact
the Swiss federal office for the environment: Municipal waste section.
 
            
    
    \hypertarget{have-a-great-day}{%
\subsubsection{\texorpdfstring{{Have a great
day}}{Have a great day}}\label{have-a-great-day}}

\textbf{This project was made possible by the Swiss federal office for
the environment.} This document originates from
https://github.com/hammerdirt-analyst/iqals all copyrights apply.
\emph{roger@hammerdirt.ch} pushed the run button on 2021-05-22.

    

    \hyperref[top]{top} \#\#\# Annex

\begin{enumerate}
\def\labelenumi{\arabic{enumi}.}
\tightlist
\item
  \hyperref[data]{Effective data}
\item
  \hyperref[popinf]{Populaion profile}
\item
  \hyperref[gps]{Survey location GPS}
\item
  \hyperref[inventory]{Inventory of all objects removed}
\end{enumerate}

\hypertarget{effective-data}{%
\subparagraph{Effective data}\label{effective-data}}

The data is submitted in two parts. If a survey is missing either part
we check the paper records and update when possible.
 
            
    
    All the surveys found a home in the dimensional dataThese dimensional
records have been dropped, there is no matching survey data:

(`luscherz-plage', `2021-01-26').

    

    \textbf{Population profile}

Survey results for each utility group, sorted by population.

    \begin{Verbatim}[commandchars=\\\{\}, frame=single, framerule=2mm, rulecolor=\color{outerrorbackground}]
\textcolor{ansi-red}{---------------------------------------------------------------------------}
\textcolor{ansi-red}{ValueError}                                Traceback (most recent call last)
\textcolor{ansi-green}{<ipython-input-37-c1c8d37b3f4c>} in \textcolor{ansi-cyan}{<module>}
\textcolor{ansi-green-intense}{\textbf{     32}}     \textcolor{ansi-green}{for} spine \textcolor{ansi-green}{in} ax\textcolor{ansi-blue}{[}an\_ax\textcolor{ansi-blue}{]}\textcolor{ansi-blue}{.}spines\textcolor{ansi-blue}{.}values\textcolor{ansi-blue}{(}\textcolor{ansi-blue}{)}\textcolor{ansi-blue}{:}
\textcolor{ansi-green-intense}{\textbf{     33}}         spine\textcolor{ansi-blue}{.}set\_visible\textcolor{ansi-blue}{(}\textcolor{ansi-green}{False}\textcolor{ansi-blue}{)}
\textcolor{ansi-green}{---> 34}\textcolor{ansi-red}{     }ax\textcolor{ansi-blue}{[}an\_ax\textcolor{ansi-blue}{]}\textcolor{ansi-blue}{.}tick\_params\textcolor{ansi-blue}{(}bottom\textcolor{ansi-blue}{=}\textcolor{ansi-green}{False}\textcolor{ansi-blue}{,} labelbottom\textcolor{ansi-blue}{=}\textcolor{ansi-green}{False}\textcolor{ansi-blue}{,} eft\textcolor{ansi-blue}{=}\textcolor{ansi-green}{False}\textcolor{ansi-blue}{,} labelleft\textcolor{ansi-blue}{=}\textcolor{ansi-green}{False}\textcolor{ansi-blue}{)}
\textcolor{ansi-green-intense}{\textbf{     35}} \textcolor{ansi-red}{\#     ax[new\_axs[0]].grid(False)}
\textcolor{ansi-green-intense}{\textbf{     36}} 

\textcolor{ansi-green}{\textasciitilde{}/anaconda3/envs/refactor\_process/lib/python3.8/site-packages/matplotlib/axes/\_base.py} in \textcolor{ansi-cyan}{tick\_params}\textcolor{ansi-blue}{(self, axis, **kwargs)}
\textcolor{ansi-green-intense}{\textbf{   3081}}             xkw\textcolor{ansi-blue}{.}pop\textcolor{ansi-blue}{(}\textcolor{ansi-blue}{'labelleft'}\textcolor{ansi-blue}{,} \textcolor{ansi-green}{None}\textcolor{ansi-blue}{)}
\textcolor{ansi-green-intense}{\textbf{   3082}}             xkw\textcolor{ansi-blue}{.}pop\textcolor{ansi-blue}{(}\textcolor{ansi-blue}{'labelright'}\textcolor{ansi-blue}{,} \textcolor{ansi-green}{None}\textcolor{ansi-blue}{)}
\textcolor{ansi-green}{-> 3083}\textcolor{ansi-red}{             }self\textcolor{ansi-blue}{.}xaxis\textcolor{ansi-blue}{.}set\_tick\_params\textcolor{ansi-blue}{(}\textcolor{ansi-blue}{**}xkw\textcolor{ansi-blue}{)}
\textcolor{ansi-green-intense}{\textbf{   3084}}         \textcolor{ansi-green}{if} axis \textcolor{ansi-green}{in} \textcolor{ansi-blue}{[}\textcolor{ansi-blue}{'y'}\textcolor{ansi-blue}{,} \textcolor{ansi-blue}{'both'}\textcolor{ansi-blue}{]}\textcolor{ansi-blue}{:}
\textcolor{ansi-green-intense}{\textbf{   3085}}             ykw \textcolor{ansi-blue}{=} dict\textcolor{ansi-blue}{(}kwargs\textcolor{ansi-blue}{)}

\textcolor{ansi-green}{\textasciitilde{}/anaconda3/envs/refactor\_process/lib/python3.8/site-packages/matplotlib/axis.py} in \textcolor{ansi-cyan}{set\_tick\_params}\textcolor{ansi-blue}{(self, which, reset, **kw)}
\textcolor{ansi-green-intense}{\textbf{    821}}         """
\textcolor{ansi-green-intense}{\textbf{    822}}         cbook\textcolor{ansi-blue}{.}\_check\_in\_list\textcolor{ansi-blue}{(}\textcolor{ansi-blue}{[}\textcolor{ansi-blue}{'major'}\textcolor{ansi-blue}{,} \textcolor{ansi-blue}{'minor'}\textcolor{ansi-blue}{,} \textcolor{ansi-blue}{'both'}\textcolor{ansi-blue}{]}\textcolor{ansi-blue}{,} which\textcolor{ansi-blue}{=}which\textcolor{ansi-blue}{)}
\textcolor{ansi-green}{--> 823}\textcolor{ansi-red}{         }kwtrans \textcolor{ansi-blue}{=} self\textcolor{ansi-blue}{.}\_translate\_tick\_kw\textcolor{ansi-blue}{(}kw\textcolor{ansi-blue}{)}
\textcolor{ansi-green-intense}{\textbf{    824}} 
\textcolor{ansi-green-intense}{\textbf{    825}}         \textcolor{ansi-red}{\# the kwargs are stored in self.\_major/minor\_tick\_kw so that any}

\textcolor{ansi-green}{\textasciitilde{}/anaconda3/envs/refactor\_process/lib/python3.8/site-packages/matplotlib/axis.py} in \textcolor{ansi-cyan}{\_translate\_tick\_kw}\textcolor{ansi-blue}{(kw)}
\textcolor{ansi-green-intense}{\textbf{    891}}         \textcolor{ansi-green}{for} key \textcolor{ansi-green}{in} kw\textcolor{ansi-blue}{:}
\textcolor{ansi-green-intense}{\textbf{    892}}             \textcolor{ansi-green}{if} key \textcolor{ansi-green}{not} \textcolor{ansi-green}{in} kwkeys\textcolor{ansi-blue}{:}
\textcolor{ansi-green}{--> 893}\textcolor{ansi-red}{                 raise ValueError(
}\textcolor{ansi-green-intense}{\textbf{    894}}                     \textcolor{ansi-blue}{"keyword \%s is not recognized; valid keywords are \%s"}
\textcolor{ansi-green-intense}{\textbf{    895}}                     \% (key, kwkeys))

\textcolor{ansi-red}{ValueError}: keyword eft is not recognized; valid keywords are ['size', 'width', 'color', 'tickdir', 'pad', 'labelsize', 'labelcolor', 'zorder', 'gridOn', 'tick1On', 'tick2On', 'label1On', 'label2On', 'length', 'direction', 'left', 'bottom', 'right', 'top', 'labelleft', 'labelbottom', 'labelright', 'labeltop', 'labelrotation', 'grid\_agg\_filter', 'grid\_alpha', 'grid\_animated', 'grid\_antialiased', 'grid\_clip\_box', 'grid\_clip\_on', 'grid\_clip\_path', 'grid\_color', 'grid\_contains', 'grid\_dash\_capstyle', 'grid\_dash\_joinstyle', 'grid\_dashes', 'grid\_data', 'grid\_drawstyle', 'grid\_figure', 'grid\_fillstyle', 'grid\_gid', 'grid\_in\_layout', 'grid\_label', 'grid\_linestyle', 'grid\_linewidth', 'grid\_marker', 'grid\_markeredgecolor', 'grid\_markeredgewidth', 'grid\_markerfacecolor', 'grid\_markerfacecoloralt', 'grid\_markersize', 'grid\_markevery', 'grid\_path\_effects', 'grid\_picker', 'grid\_pickradius', 'grid\_rasterized', 'grid\_sketch\_params', 'grid\_snap', 'grid\_solid\_capstyle', 'grid\_solid\_joinstyle', 'grid\_transform', 'grid\_url', 'grid\_visible', 'grid\_xdata', 'grid\_ydata', 'grid\_zorder', 'grid\_aa', 'grid\_c', 'grid\_ds', 'grid\_ls', 'grid\_lw', 'grid\_mec', 'grid\_mew', 'grid\_mfc', 'grid\_mfcalt', 'grid\_ms']
    \end{Verbatim}

    \begin{center}
    \adjustimage{max size={0.9\linewidth}{0.9\paperheight}}{feature_report_files/feature_report_52_1.png}
    \end{center}
    { \hspace*{\fill} \\}
    
    \hyperref[annex]{Annex}

\textbf{Survey locations:}

    \hyperref[annex]{Annex} \#\#\#\# Inventory of all items


    % Add a bibliography block to the postdoc
    
    
    
\end{document}
